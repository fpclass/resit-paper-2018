\allowdisplaybreaks
%%% Question 4 - - - - - - - - - - - - - - - - - - - - - - - - - - - - - -
\question This question is about equational reasoning.

A compiler is said to be correct if the result of interpreting a program according to its semantics is the same as compiling the program and then running it. Consider the following data type to represent a small expression language:
\begin{verbatim}
data Expr = Val Int | Add Expr Expr
\end{verbatim}
\begin{parts}
\part[2] Define a function
\begin{center}
	\texttt{eval~::~Expr -> Int}
\end{center}
which interprets expressions. For example, \texttt{eval (Add (Val 4) (Add (Val 8) (Val 23)))} should evaluate to \texttt{35}. \droppoints
\begin{solution}
\emph{Application.} 
\begin{verbatim}
eval (Val n)   = n
eval (Add l r) = eval l + eval r
\end{verbatim}
\end{solution}
\part[4] To compile this expression language, we will use a simple, stack-based machine whose instruction set is represented by the following data type:
\begin{verbatim}
data Instr = PUSH Int | ADD
\end{verbatim}
A program is then a list of values of type \texttt{Instr} and the machine's stack is just a list of values of type \texttt{Int}:
\begin{verbatim}
type Program = [Instr]
type Stack   = [Int]
\end{verbatim}
Define a function
\begin{center}
	\texttt{exec~::~Program -> Stack -> Stack}
\end{center}
which executes a program. For example, \texttt{exec [PUSH 4, PUSH 8, PUSH 32, ADD, ADD] []} should evaluate to \texttt{[35]}. \droppoints 
\begin{solution}
\emph{Application.} 
\begin{verbatim}
exec [] s              = s 
exec (PUSH n : p) s    = exec p (n:s)
exec (ADD : p) (x:y:s) = exec p (x+y:s)
\end{verbatim}
\end{solution}
\part[4] Finally, we can write a compiler from expressions of type \texttt{Expr} to programs of type \texttt{Program}:
\begin{center}
	\texttt{comp~::~Expr -> Program}
\end{center}
Define this function so that, for example, \texttt{comp (Add (Val 4) (Add (Val 8) (Val 23)))} evaluates to \texttt{[PUSH 4, PUSH 8, PUSH 32, ADD, ADD]}. \droppoints 
\begin{solution}
\emph{Application.}
\begin{verbatim}
comp (Var n)   = [PUSH n]
comp (Add l r) = comp r ++ comp l ++ [ADD]
\end{verbatim}
\end{solution}
\part The following property states that executing a program \texttt{xs} followed by a program \texttt{ys} with an initial stack \texttt{s} yields the same result as executing \texttt{xs} with an initial stack \texttt{s} and then executing \texttt{ys} with the resulting stack:
\begin{displaymath}
\mathit{exec}~(xs \append ys)~s = \mathit{exec}~ys~(\mathit{exec}~xs~s)
\end{displaymath}
This proof will be by induction on \texttt{xs}. 
\begin{subparts}
\subpart[1] Prove the base case for \texttt{[]}. \droppoints 
\begin{solution}
\emph{Application.}
\begin{verbatim}
	Base case of exec ([] ++ ys) s = exec ys (exec [] s)
	
	exec ys (exec [] s)
	= {applying exec}
	exec ys s 
	= {unapplying ++}
	exec ([] ++ ys) s
\end{verbatim}
\end{solution}
\subpart[3] Prove the inductive case for \texttt{x:xs} and \texttt{x=PUSH n} for all \texttt{n}. \droppoints
\begin{solution}
\emph{Application.}
\begin{verbatim}
Assume exec (xs ++ ys) s = exec ys (exec xs s).
Inductive case of exec ((x:xs) ++ ys) s = exec ys (exec (x:xs) s)

Case analysis on x.
Case PUSH n:

exec ys (exec (PUSH n:xs) s)
= {applying exec}
exec ys (exec xs (n:s))
= {induction hypothesis}
exec (xs ++ ys) (n:s)
= {unapplying exec}
exec (PUSH n : (xs ++ ys))
= {unapplying ++}
exec ((PUSH n:xs) ++ ys) s
\end{verbatim}
\end{solution}
\subpart[3] Prove the inductive case for \texttt{x:xs} and \texttt{x=ADD}. You may also assume that the stack contains at least two values \texttt{s=x:y:zs}. \droppoints
\begin{solution}
	\emph{Application.}
	\begin{verbatim}
	Assume exec (xs ++ ys) s = exec ys (exec xs s).
	Inductive case of exec ((x:xs) ++ ys) s = exec ys (exec (x:xs) s)
	
	Case analysis on x.	
	Case ADD:
	
	exec ys (exec (ADD:xs) (x:y:zs))
	= {applying exec}
	exec ys (exec xs (x+y:zs))
	= {induction hypothesis}
	exec (xs ++ ys) (x+y:zs)
	= {unapplying exec}
	exec (ADD : (xs ++ ys)) (x+y:zs)
	= {unapplying ++}
	exec ((ADD : xs) ++ ys) (x+y:zs)
	\end{verbatim}
\end{solution}
\end{subparts}
\part[8] Prove the following compiler correctness theorem: 
	\begin{displaymath}
	\mathit{eval}~e : s = \mathit{exec}~(\mathit{comp}~e)~s
	\end{displaymath} \droppoints
\begin{solution}
\emph{Application.}
\begin{verbatim}
Proof is by induction on e.

Base case of eval (Var n) : s = exec (comp (Var n)) s

exec (comp (Var n)) s
= {applying comp}
exec [PUSH n] s 
= {applying exec}
n:s
= {unapplying eval}
eval (Var n) : s

Assume (1) eval l : s = exec (comp l) s 
       (2) eval r : s = exec (comp r) s
       
exec (comp (Add l r)) s 
= {applying comp}
exec (comp r ++ comp l ++ [ADD]) s
= {lemma}
exec [ADD] (exec (comp l) (exec (comp r) s))
= {induction hypothesis}
exec [ADD] (exec (comp l) (eval r : s))
= {induction hypothesis}
exec [ADD] (eval l : (eval r : s))
= {applying exec}
exec [] (eval l + eval r : s)
= {applying exec}
eval l + eval r : s
= {unapplying eval }
eval (Add l r) : s
\end{verbatim}
\end{solution}
\end{parts}
