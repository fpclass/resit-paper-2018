
%%% Question 5 - - - - - - - - - - - - - - - - - - - - - - - - - - - - - -
\question This question is about monads.
\begin{parts} 
\part[4] Consider the following expression language:
\begin{verbatim}
data Expr a = Var a | App (Expr a) (Expr a)
\end{verbatim}
This type is a monad. Define a suitable instance of the \texttt{Monad} type class for the \texttt{Expr} type. \droppoints 
\begin{solution} \emph{Application.}
\begin{verbatim}
instance Monad Expr where 
    return = Var
    
    Var x >>= f = f x
    (App l r) >>= f = App (l >>= f) (r >>= f)
\end{verbatim}
\end{solution}
\part[2] With the help of a suitable example, explain the effect of the \texttt{Expr} monad. \droppoints 
\begin{solution}
\emph{Comprehension. 1 mark for describing the effect and 1 mark for an illustrative example.} The effect is substitution. For example, \texttt{App (Var "f") (Var "x") >>= \textbackslash x -> Var ("n"++x)} evaluates to \texttt{App (Var "nf") (Var "nx")}.
\end{solution}
\part[10] Prove that your instance of the \texttt{Monad} type class for the \texttt{Expr} type obeys the monad laws. \droppoints
\begin{solution}
	\emph{Application.} We can prove left identity through simple equational reasoning to rewrite one side of the equation to the other:
	\begin{verbatim}
	return x >>= f 
	= { Definition of return }
	Var x >>= f 
	= { Definition of bind }
	f x
	\end{verbatim}
	The proof for right identity is by induction on $m$. In the base case we have that $m = \mathit{Var~a}$ for all $a$:
	\begin{verbatim}
	Var a >>= return 
	= { applying bind }
	return a
	= { applying return }
	Var a
	\end{verbatim}
	In the recursive case, $m = \mathit{App}~l~r$ for all $l$ and all $r$:
	\begin{verbatim}
	App l r >>= return 
	= { applying bind }
	App (l >>= return) (r >>= return)
	= { induction hypotheses }
	App l r
	\end{verbatim}
	The proof for associativity is also by induction on $m$. In the base case we again have that $m = \mathit{Var~a}$ for all $a$:
	\begin{verbatim}
	(Var a >>= f) >>= g
	= { applying bind }
	(f a) >>= g 
	= { eta expansion }
	(\x -> f x >>= g) a
	= { unapplying bind }
	Var a >>= (\x -> f x >>= g)
	\end{verbatim}
	In the recursive case, $m = \mathit{App}~l~r$ for all $l$ and all $r$: \allowdisplaybreaks
	\begin{verbatim}
	(App l r >>= f) >>= g 
	= { applying bind }
	App (l >>= f) (r >>= f) >>= g
	= { applying bind }
	App ((l >>= f) >>= g) ((r >>= f) >>= g)
	= { induction hypotheses }
	App (l >>= (\x -> f x >>= g)) (r >>= (\x -> f x >>= g))
	= { unapplying bind }
	App l r >>= (\x -> f x >>= g)
	\end{verbatim}
\end{solution}
\part[9] Monad transformers are monads which are defined in terms of other, arbitrary monads. A suitable type for a state monad transformer is:
\begin{verbatim}
data StateT s m a = StT (s -> m (a,s))
\end{verbatim}
This type is a monad. Complete the definition of the \texttt{Monad} instance by implementing \texttt{return} and \texttt{>{}>=}: 
\begin{verbatim}
instance Monad m => Monad (StateT s m) where 
  ???
\end{verbatim}
\emph{Hint}: make use of the fact that \texttt{m} is required to be an instance of \texttt{Monad}. \droppoints 
\begin{solution}
\emph{Application. 3 marks for \texttt{return} and 6 marks for \texttt{>{}>=}.}
\begin{verbatim}
return x = StT (\s -> return (x,s))

(StT m) >>= f = StT (\s -> m s >>= \(x,s') -> 
                     let (StT m') = f x in m' s')
\end{verbatim}
\end{solution}
\end{parts}
