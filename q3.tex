
%%% Question 3 - - - - - - - - - - - - - - - - - - - - - - - - - - - - - -
\question This question is about user-defined types and type classes.
\begin{parts}
\part A binary tree is a tree data structure where every node has at most two children. In Haskell, we could represent binary trees using the following data type where elements are stored in nodes:
\begin{verbatim}
data BinTree a = Leaf | Node (BinTree a) a (BinTree a)
\end{verbatim} 
As an example of a value of this type, consider the following definition for a binary tree containing the numbers 1, 2, and 3 as elements from left to right:
\begin{verbatim}
tree :: BinTree Int 
tree = Node (Node Leaf 1 Leaf) 2 (Node Leaf 3 Leaf)
\end{verbatim}
\begin{subparts}
\subpart[2] What is the type of the \texttt{Node} constructor? \droppoints 
\begin{solution} \emph{Comprehension.}
\begin{verbatim}
Node :: BinTree a -> a -> BinTree a -> BinTree a
\end{verbatim}
\end{solution}
\subpart[6] Define a function
\begin{center}
	\texttt{inorder~::~BinTree a -> [a]}
\end{center}
which performs an inorder traversal of the binary tree. For example, \texttt{inorder tree} should evaluate to \texttt{[1,2,3]}. Additionally, your function should run in linear time with respect to the size of the input. You should not assume that the compiler will perform any optimisations for you. \droppoints
\begin{solution}
\begin{verbatim}
inorder :: BinTree a -> [a]
inorder t = go t []
  where 
    go Leaf xs         = xs 
    go (Node l x r) xs = go l (x : go r xs)
\end{verbatim}
\end{solution}
\subpart[4] Define a function 
\begin{center}
	\texttt{traversePO~::~Monad m => (a -> m b) -> BinTree a -> m [b]}
\end{center}
which performs a postorder traversal of the binary tree and applies a function of type \texttt{a -> m b} to every element. For example, \texttt{traversePO  print tree} would evaluate to \texttt{[(), (), ()]} and write the following three lines to the standard output:

\texttt{1}\\
\texttt{3}\\
\texttt{2}  \droppoints 
\begin{solution}
\emph{Application.}
\begin{verbatim}
traversePO :: Monad m => (a -> m b) -> BinTree a -> m [b]
traversePO _ Leaf         = return []
traversePO f (Node l x r) = do 
  ls <- traversePO f l 
  rs <- traversePO f r
  y  <- f x
  return (ls ++ rs ++ [y])
\end{verbatim}
\end{solution}
\subpart[4] A binary tree containing pairs of keys and values can be used to represent a key-value store. Define a function 
\begin{center}
	\texttt{insert :: Ord k => k -> v -> BinTree (k,v) -> BinTree (k,v)}
\end{center}
which inserts a value of type \texttt{v} with a key of type \texttt{k} into a key-value store that is represented by values of type \texttt{BinTree (k,v)}. The function should behave as follows: if no element with the specified key exists, it should be added in the correct place. For example, suppose we have an existing key-value store:
\begin{verbatim}
store :: BinTree (Int, String)
store = Node (Node Leaf (118, "Java") Leaf) 
             (256, "FP" ) 
             (Node Leaf (263, "Cyber Security") Leaf)
\end{verbatim}
Then \texttt{insert 255 "AI" store} should evaluate to:
\begin{verbatim}
Node (Node Leaf (118, "Java") (Node Leaf (255, "AI") Leaf )) 
     (256, "FP") 
     (Node Leaf (263, "Cyber Security") Leaf)
\end{verbatim}
If a key already exists in the key-value store, then its value should be overwritten with the specified one. \droppoints 
\begin{solution}
\emph{Application.}
\begin{verbatim}
insert k v Leaf = Node Leaf (k,v) Leaf 
insert k v (Node l (k',x) r)
  | k == k' = Node l (k,v) r
  | k <  k' = Node (insert k v l) (k',x) r
  | otherwise = Node l (k',x) (insert k v r)
\end{verbatim}
\end{solution}
\pagebreak
\subpart[4] Define a function 
\begin{center}
	\texttt{lookup :: Ord k => k -> BinTree (k,v) -> Maybe v}
\end{center}
which tries to look up the value corresponding to some key of type \texttt{k} in a key-value store. If the key exists, its value should be returned wrapped in \texttt{Just}. Otherwise, the function should return \texttt{Nothing}. \droppoints 
\begin{solution}
\emph{Application.}
\begin{verbatim}
lookup k Leaf = Nothing 
lookup k (Node l (k',v) r)
  | k == k' = Just v 
  | k <  k' = lookup k l 
  | otherwise = lookup k r
\end{verbatim}
\end{solution}
\end{subparts}
\part[5] The \texttt{BinTree} type is a foldable data structure. Define a suitable instance of \texttt{Foldable} for this type by implementing \texttt{foldr}. \droppoints 
\begin{solution}
\emph{Application.}
\begin{verbatim}
instance Foldable BinTree where 
    foldr f v Leaf         = v 
    foldr f v (Node l x r) = f x (foldr f (foldr f z r) l)
\end{verbatim}
\end{solution}
\end{parts}
