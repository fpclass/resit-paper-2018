%%% Question 2 - - - - - - - - - - - - - - - - - - - - - - - - - - - - - -
\question This question is about recursive and higher-order functions.

In the \emph{countdown problem} you are given a list of numbers, the standard arithmetic operators (\texttt{+}, \texttt{-}, \texttt{*}, \texttt{/}), and a target number. The goal is to construct an arithmetic expression using the given list of numbers and the standard arithmetic operators which evaluates to the target number. Each number may only be used once and values, including intermediate results, can never be negative. For example, if the list of numbers is \texttt{[1, 3, 7, 10, 25, 50]} and the target is \texttt{765}, then a valid solution to the countdown problem is \texttt{(1 + 50) * (25 - 10)}. We represent operators and expressions as follows:
\begin{verbatim}
data Op   = Add | Sub | Mul | Div
data Expr = Val Int | App Op Expr Expr
\end{verbatim}
\begin{parts}
\part[3] With the help of \texttt{split}, define a function 
\begin{center}
	\texttt{split~::~[a] -> [([a],[a])]}
\end{center}
which returns a list of pairs describing all possible ways to split a list into two non-empty ones. For example, \texttt{split [1,2,3,4]} should evaluate to \texttt{[([1],[2,3,4]), ([1,2],[3,4]), ([1,2,3],[4])]}. \droppoints 
\begin{solution}
\emph{Application.}
\begin{verbatim}
split [] = []
split [_] = []
split (x:xs) = ([x],xs) : [(x:ls,rs) | (ls,rs) <- split xs]
\end{verbatim}
\end{solution}
\part[5] With the help of \texttt{split}, define a function 
\begin{center}
	\texttt{exprs~::~[Int] -> [Expr]}
\end{center}
which generates all expressions for a list of numbers. All of the numbers from the list should be used and in the order provided. For example, \texttt{exprs [1,2]} should evaluate to \texttt{[App Add (Val 1) (Val 2), App Sub (Val 1) (Val 2), App Mul (Val 1) (Val 2), App Div (Val 1) (Val 2)]}.\\ \emph{Hint:} for longer lists, expressions will have to be nested further. \droppoints
\begin{solution}
\emph{Application}
\begin{verbatim}
exprs [] = []
exprs [n] = [Val n]
exprs ns = [e | (ls,rs) <- split ns,
                l <- exprs ls,
                r <- exprs rs,
                e <- [App o l r | o <- [Add, Sub, Mul, Div]]]
\end{verbatim}
\end{solution}
\part[5] Define a function 
\begin{center}
	\texttt{eval~::~Expr -> [Int]}
\end{center}
which evaluates an expression. For example, \texttt{eval (App Div (Val 4) (Val 3)} should evaluate to \texttt{[1]}. This function should evaluate to \texttt{[]} if the expression cannot be evaluated. \droppoints 
\begin{solution}
\emph{Application.}
\begin{verbatim}
eval (Val n) = [n]
eval (App o l r) = do 
  x <- eval l
  y <- eval r 
  [apply o x y]
  
apply Add x y = x + y
apply Sub x y = x - y
apply Mul x y = x * y
apply Div x y = x `div`y
\end{verbatim}
\end{solution}
\part[3] Define a function 
\begin{center}
	\texttt{permutations~::~Eq a => [a] -> [[a]]}
\end{center}
which calculates all permutations of a list. For example, \texttt{permutations [1,2,3]} should evaluate to \texttt{[[1,2,3],[2,1,3],[2,3,1],[1,3,2],[3,1,2], [3,2,1]]}. \droppoints 
\begin{solution}
\emph{Bookwork.}
\begin{verbatim}
permutations :: Eq a => [a] -> [[a]]
permutations [] = [[]]
permutations xs = [x : ys | x <- xs, ys <- permutations (delete x xs)]
\end{verbatim}
\end{solution}
\part[3] Define a function
\begin{center}
	\texttt{subsequences~::~[a] -> [[a]]}
\end{center}
which calculates all subsequences of a list. For example, \texttt{subsequences [1,2,3]} should evaluate to \texttt{[[],[3],[2],[2,3],[1],[1,3],[1,2],}\\\texttt{[1,2,3]]}. \droppoints 
\begin{solution}
\emph{Bookwork.}
\begin{verbatim}
subsequences :: [a] -> [[a]]
subsequences [] = [[]]
subsequences (x:xs) = ys ++ map (x:) ys
where ys = subsequences xs
\end{verbatim}
\end{solution}
\pagebreak
\part[2] With the help of \texttt{subsequences} and \texttt{permutations}, define a function 
\begin{center}
	\texttt{choices :: [a] -> [[a]]}
\end{center}
which generates all permutations of subsequences of a list. For example, \texttt{choices [1,2,3]} should evaluate to \texttt{[[], [3], [2], [2, 3], [3, 2], [1],}\\\texttt{[1, 3], [3, 1], [1, 2], [2, 1], [1, 2, 3], [2, 1, 3],}\\ \texttt{[2, 3, 1],[1, 3, 2], [3, 1, 2], [3, 2, 1]]}. \droppoints 
\begin{solution}
\emph{Application.}
\begin{verbatim}
choices = concat . map permutations . subsequences
\end{verbatim}
\end{solution}
\part[4] With the help of \texttt{choices}, \texttt{exprs}, and \texttt{eval}, define a function 
\begin{center}
	\texttt{countdown~::~[Int] -> Int -> [Expr]}
\end{center}
which solves the countdown problem for a given list of numbers and a target number. \droppoints 
\begin{solution}
\emph{Application.}
\begin{verbatim}
countdown ns n = [e | ns' <- chopices ns,
                      e <- exprs ns',
                      eval e == [n]]
\end{verbatim}
\end{solution}
\end{parts}
